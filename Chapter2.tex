%%%%%%%%%%%%%%%%%%%%%%%%%%%%%%%%%%%%%%%%%%%%%%%%%%%%%%%%%%%%
% Motivación de la tesis
%%%%%%%%%%%%%%%%%%%%%%%%%%%%%%%%%%%%%%%%%%%%%%%%%%%%%%%%%%%%
\chapter{Sobre guías de ondas}
%\addcontentsline{toc}{chapter}{\ Thorough Derivation of the Rabi Equation}

%%%%%%%%%%%%%%%%%%%%%%%%%%%%%%%%%%%%%%%%%%%%%%%%
\subsection{Helm}
\begin{subequations}\label{Maxwell}
	\begin{align}
	\nabla \cdot E = 0,\label{Primera}\\
	\nabla \times = - \partial_{t} B,\label{segunda}\\
	\nabla \cdot B = 0,\label{tercera}\\
	\nabla \times B=  \frac{1}{v^2} \partial_{t} E,\label{cuarta}
	\end{align}
\end{subequations}
\begin{subequations}\label{Camposincidentes}
	\begin{align}
	\hat{E_{0}} = E_{x}\hat{x} + E_{y}\hat{y}+E_{z}\hat{z} \\
	\hat{B_{0}} = B_{x}\hat{x} + B_{y}\hat{y}+B_{z}\hat{z}
	\end{align}
\end{subequations}
\begin{subequations}\label{Camposexplicitos}
	\begin{align}
	\partial_{y}E_{x}-\partial_{x}E_{y}= i w B_{z} \\
	i k E_{x}-\partial_{x}E_{z}= i w B_{y} \\
	\partial_{y}E_{z}-i k E_{y}= i w B_{x} \\
	\partial_{x}B_{z}-\partial_{y}B_{x}= \frac{-iw E_{Z} }{v^2} \\
	i k B_{x}-\partial_{x}B_{z}= \frac{-iw E_{y} }{v^2} \\
	\partial_{y}B_{z}- i k B_{y}= \frac{-iw E_{x} }{v^2} 
	\end{align}
\end{subequations}
%Parte 3 waveguide fisico
Operando entre ellas se puede llegar ha
\begin{subequations}\label{Ecuacionesdecampo}
	\begin{align}
	E_{x} = \frac{i}{(\frac{w}{v})^2 - k^2} (k \partial_{x}E_{z}+w \partial_{y} B_{z}) \\
	E_{y} = \frac{i}{(\frac{w}{v})^2 - k^2} (k \partial_{y}E_{z}-w \partial_{x} B_{z}) \\
	B_{x} = \frac{i}{(\frac{w}{v})^2 - k^2} (k \partial_{x}B_{z}- \frac{w}{v^2} \partial_{y} E_{z}) \\
	B_{y} = \frac{i}{(\frac{w}{v})^2 - k^2} (k \partial_{y}B_{z}- \frac{w}{v^2} \partial_{x} E_{z}) 
	\end{align}
\end{subequations}
De este procedimiento se obtienen  
\begin{subequations}\label{Helmholtz}
	\begin{align}
	[\partial_{x}^2+\partial_{y}^2+(\frac{w}{v})^2-k^2] E_{z}=0\\
	[\partial_{x}^2+\partial_{y}^2+(\frac{w}{v})^2-k^2] B_{z}=0
	\end{align}
\end{subequations}
%%%%%%%%%%%%%%%%%%%%%%%%%%%%%%%%%%%%%%%%%%%%%%%%%%%%%%%%%%%%%%%%%%%%%%%%%%%%%%%%%%%%%%%%%%%%%%%%%%
\subsection{Guías en fibras metálicas}
\begin{subequations} \label{mawxellmetalicas}
	\begin{eqnarray}
	\nabla \cdot \mathbf{E} &=& 0 ,\\
	\nabla \times E &=& \partial_{t}B, \\
	\nabla \cdot B &=& 0, \\ 
	\nabla \times B &=& \mu \epsilon E + \mu \sigma E  
	\end{eqnarray}
\end{subequations}
Operando el rotaciones en la ultima ecuacion y usando la identidad del triple producto vectorial obtenemos
\begin{subequations}\label{potencialmetalico}
	\begin{align}
	\nabla^2 E=\mu \epsilon \partial_{t}^2E+\mu\sigma \partial_{t} E \\
	\nabla^2 B=\mu \epsilon \partial_{t}^2B+\mu\sigma \partial_{t} B 
	\end{align}
\end{subequations}
Utilizando la formulación imaginaria de los campos
\begin{eqnarray}
\mathbf{E}(\vec{r},t) &=& E_{0}(\vec{r}_{\perp}) e^{\hat{k}z-wt} ; \\
\mathbf{B}(\vec{r},t) &=& B_{0}(\vec{r}_{\perp}) e^{\hat{k}z-wt} 
\end{eqnarray}
Operando con ellas y usando las ecuaciones~\ref{potencialmetalico}, obtenemos una ecuacion para
el numero de onda
\begin{equation}
k^2=\mu \epsilon w^2 + i \mu \sigma w, \qquad k \in \mathbb{C}, \quad \mu, \epsilon, \sigma \in \mathbb{R}
\end{equation}
Quitando las dependecia de la frecuencia $w$ 
\begin{subequations}
	\begin{eqnarray}
	(k+i\kappa)^2 &=& k^2-\kappa^2+2ik\kappa=\mu\epsilon w^2+i\mu\sigma w \nonumber \\
	k^2 - \kappa^2 &=& \mu \epsilon w^2, \\ 
	2 k \kappa &=& \mu \sigma w \nonumber \\
	\end{eqnarray}
\end{subequations}
por lo que 
\begin{eqnarray}
\kappa &=& \frac{\mu \epsilon w}{2 k}
\end{eqnarray}


Metiendo las equivalencias obtenidas a las ecuaciones originales 
\begin{subequations}
	\begin{eqnarray}
	k^4 - \frac{\mu^2 \sigma^2 w^2}{4} - \mu \epsilon w^2=0 \nonumber\\
	\end{eqnarray}
\end{subequations}
Completando cuadrados llegamos a 
\begin{subequations}
	\begin{eqnarray}
	k= w \sqrt{\frac{\epsilon \mu}{2}} \left[\sqrt{1+\left(\frac{\sigma}{\epsilon w} \right)^2}+1 \right]^\frac{1}{2}
	\end{eqnarray}
\end{subequations}
como $\kappa^2 = k^2 - \mu \epsilon w^2$ luego
\begin{subequations}
	\begin{eqnarray}
	\kappa = w \sqrt{\frac{\epsilon \mu}{2}} \left[\sqrt{1 + \left(\frac{\sigma}{\epsilon w} \right)^2}-1\right]^\frac{1}{2}
	\end{eqnarray}
\end{subequations}
%%%%%%%%%%%%%%%%%%%%%%%%%%%%%%%%%%%%%%%%%%%%%%%%%%%%%%%%%%%%%%%%%%%%%%%%%%%%%%%%%%%%%%%%%%%%%%%%%%
\subsubsection{Guías cuadradas}
\begin{subequations}
	\begin{align}
	B_{z}(x,y)=X(x)Y(y) \nonumber \\
	Y d_{x}^2 X+X d_{y}^2Y + [(\frac{w}{v})^2-k^2]XY=0 \nonumber \\
	\frac{1}{X} d_{x}^2 X=-k_{x}^2 ; \frac{1}{Y}d_{y}^2 Y = -K_{y}^2 \nonumber \\
	-k_{x}^2 - k_{y}^2+(\frac{w}{v})^2 - k^2 = 0 \nonumber\\
	\text{Proponiendo } \nonumber \\
	X(x)=A \sin (K_{x} x) + B \cos(k_{x}x)  \\
	\text{metendielo en las ecuaciones originales} \nonumber \\
	B_{z}=B_{0} \cos(m \pi \frac{x}{a}) \cos(n \pi \frac{y}{a}) \\
	k = \sqrt{(\frac{w}{v})^2-\pi^2 [(\frac{m}{a})^2+(\frac{n}{b})^2]} 
	\end{align}
\end{subequations}
%%%%%%%%%%%%%%%%%%%%%%%%%%%%%%%%%%%%%%%%%%%%%%%%%%%%%%%%%%%%%%%%%%%%%%%%%%%%%%%%%%%%%%%%%%%%%%%%%%
\subsubsection{Guías esféricas}

%%%%%%%%%%%%%%%%%%%%%%%%%%%%%%%%%%%%%%%%%%%%%%%%%%%%%%%%%%%%%%%%%%%%%%%%%%%%%%%%%%%%%%%%%%%%%%%%%%
\subsection{Guías dieléctricas}
Que son las ecuaciones de Helmholtz en dos dimensiones. Estas ecuaciones pueden ser generalizadas a
\begin{subequations}
	\begin{align}
	B_{t}=\frac{}{\mu \epsilon\frac{w^2}{c^2}-k^2}[D_{t} (d_{z} B{z})+i\mu\epsilon\frac{w}{v}e_{3}\wedge D_{t}E_{z} \\
	E_{t}=\frac{}{\mu \epsilon\frac{w^2}{c^2}-k^2}[D_{t} (d_{z} E{z})-i\frac{w}{v}e_{3}\wedge D_{t}B_{z} 
	\end{align}
	\end{subequations}
Donde $t$ es la componente transversal
\begin{subequations}
	\begin{align}
	B_{t}=\frac{}{\gamma^2} [D_{t}(d_{z}B_{z})+i\mu \epsilon \frac{w}{v} e_{3}\wedge D_{t}E_{z}]
	\end{align}
\end{subequations}
Desarrollando todas las componentes implicada
\begin{subequations}
	\begin{align}
B_{rho} = \frac{ik}{\gamma^2}d_{rho}B_{z} ; B_{\phi} = \frac{i \epsilon w}{\gamma^2 v} d_{\rho} E_{z}\\
E_{\phi} = \frac{-w}{v k }B_{\rho} ; E_{\rho} = \frac{v k}{\epsilon w} B_{\phi}
	\end{align}
\end{subequations}
Con soluciones 
\begin{subequations}
	\begin{gather}
		B_{z} = J_{0}(\gamma \rho) para \rho \leq a \nonumber \\
		B_{\rho } = \frac{-i k}{\gamma} J_{1} (\gamma \rho) \nonumber \\
		E_{\phi} =
	\end{gather}
\end{subequations}
%%%%%%%%%%%%%%%%%%%%%%%%%%%%%%%%%%%%%%%%%%%%%%%%%%%%%%%%%%%%%%%%%%%%%%%%%%%%%%%%%%%%%%%%%%%%%%%%%%
\subsubsection{Guías cuadradas}
%%%%%%%%%%%%%%%%%%%%%%%%%%%%%%%%%%%%%%%%%%%%%%%%%%%%%%%%%%%%%%%%%%%%%%%%%%%%%%%%%%%%%%%%%%%%%%%%%%
\subsubsection{Guías esféricas}
\begin{subequations}
	\begin{align}
	\gamma^2 =  \mu_{1} \epsilon_{1} \frac{w^2}{v^2} - k^2   && \text{  Dentro}\\
	\textcolor{red}{\beta^2 = k^2 - \mu_{0} \epsilon_{0} \frac{w^2}{v^2}}   && \text{ Fuera (onda desvaneciente)}
	\end{align}
\end{subequations}
La ecuacion de Helmholtz para variación azimutal toma la forma
\begin{subequations}
	\begin{align}
	(d_{\rho}^2+\rho^{-1} d_{\rho}+\gamma^2) \psi = 0\\
	(d_{\rho}^2+\rho^{-1} d_{\rho}-\beta^2) \psi = 0
	\end{align}
\end{subequations}
Que es la ecuación diferencial de Bessel y cuya solución toma la forma
\begin{equation} \label{Soluciones metalico bessel}
\begin{split}
\psi  & = \bigg \{ J_{0} (\gamma \rho), \rho \leq a\\
&   A k_{0} (\beta \rho), \rho > a
\end{split}
\end{equation}
%%%%%%%%%%%%%%%%%%%%%%%%%%%%%%%%%%%%%%%%%%%%%%%%%%%%%%%%%%%%%%%%%%%%%%%%%%%%%%%%%%%%%%%%%%%%%%%%%%
\subsection{Acoplamientos}