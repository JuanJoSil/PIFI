%%%%%%%%%%%%%%%%%%%%%%%%%%%%%%%%%%%%%%%%%%%%%%%%%%%%%%%%%%%%
% Motivación de la tesis
%%%%%%%%%%%%%%%%%%%%%%%%%%%%%%%%%%%%%%%%%%%%%%%%%%%%%%%%%%%%
\chapter{Sobre guías de ondas}
%\addcontentsline{toc}{chapter}{\ Thorough Derivation of the Rabi Equation}

%%%%%%%%%%%%%%%%%%%%%%%%%%%%%%%%%%%%%%%%%%%%%%%%
\subsection{Helm}
\begin{subequations}\label{Maxwell}
	\begin{align}
	\nabla \cdot E = 0,\label{Primera}\\
	\nabla \times = - d_{t} B,\label{segunda}\\
	\nabla \cdot B = 0,\label{tercera}\\
	\nabla \times B=  \frac{_{t} E}{v^2},\label{cuarta}
	\end{align}
\end{subequations}
\begin{subequations}\label{Camposincidentes}
	\begin{align}
\hat{E_{0}} = E_{x}\hat{x} + E_{y}\hat{y}+E_{z}\hat{z} \\
\hat{B_{0}} = B_{x}\hat{x} + B_{y}\hat{y}+B_{z}\hat{z}
	\end{align}
\end{subequations}
\begin{subequations}\label{Camposexplicitos}
	\begin{align}
d_{y}E_{x}-d_{x}E_{y}= i w B_{z} \\
i k E_{x}-d_{x}E_{z}= i w B_{y} \\
d_{y}E_{z}-i k E_{y}= i w B_{x} \\
d_{x}B_{z}-d_{y}B_{x}= \frac{-iw E_{Z} }{v^2} \\
i k B_{x}-d_{x}B_{z}= \frac{-iw E_{y} }{v^2} \\
d_{y}B_{z}- i k B_{y}= \frac{-iw E_{x} }{v^2} 
	\end{align}
\end{subequations}
%Parte 3 waveguide fisico
Operando entre ellas se puede llegar ha
\begin{subequations}\label{Ecuacionesdecampo}
	\begin{align}
	E_{x} = \frac{i}{(\frac{w}{v})^2 - k^2} (k d_{x}E_{z}+w d_{y} B_{z}) \\
	E_{y} = \frac{i}{(\frac{w}{v})^2 - k^2} (k d_{y}E_{z}-w d_{x} B_{z}) \\
	B_{x} = \frac{i}{(\frac{w}{v})^2 - k^2} (k d_{x}B_{z}- \frac{w}{v^2} d_{y} E_{z}) \\
	B_{y} = \frac{i}{(\frac{w}{v})^2 - k^2} (k d_{y}B_{z}- \frac{w}{v^2} d_{x} E_{z}) 
	\end{align}
\end{subequations}
De este procedimiento se obtienen  
\begin{subequations}\label{Helmholtz}
	\begin{align}
[d_{x}^2+d_{y}^2+(\frac{w}{v})^2-k^2] E_{z}=0\\
[d_{x}^2+d_{y}^2+(\frac{w}{v})^2-k^2] B_{z}=0
	\end{align}
\end{subequations}
%%%%%%%%%%%%%%%%%%%%%%%%%%%%%%%%%%%%%%%%%%%%%%%%%%%%%%%%%%%%%%%%%%%%%%%%%%%%%%%%%%%%%%%%%%%%%%%%%%
\subsection{Guías en fibras metálicas}
\begin{subequations}
	\begin{align} \label{mawxellmetalicas}
	\nabla \cdot E= 0 , \nabla \times E = d_{t}B\\
	\nabla \cdot B=0, \nabla \times B = \mu \epsilon E + \mu \sigma E  
	\end{align}
	\end{subequations}
Operando el rotaciones en la ultima ecuacion y usando la identidad del triple producto vectorial obtenemos
\begin{subequations}\label{potencialmetalico}
\begin{align}
	\nabla^2 E=\mu \epsilon d_{t}^2E+\mu\sigma d_{t} E \\
	\nabla^2 B=\mu \epsilon d_{t}^2B+\mu\sigma d_{t} B 
\end{align}
\end{subequations}
Utilizando la formulación imaginaria de los campos
\begin{equation}
	\widehat{E}(z,t)=\widehat{E}_{0} e^{\hat{k}z-wt} ; \widehat{B}(z,t)=\widehat{B}_{0} e^{\hat{k}z-wt} 
\end{equation}
Operando con ellas y usando las ecuaciones~\ref{potencialmetalico}, obtenemos una ecuacion para
el numero de onda
\begin{equation}
\widehat{k}^2=\mu \epsilon w^2 + i \mu \sigma w
\end{equation}
Quitando las dependecia de la frecuencia $w$ 
\begin{subequations}
	\begin{gather}
	(k+i\kappa)^2=k^2-\kappa^2+2ik\kappa=\mu\epsilon w^2+i\mu\sigma w \nonumber \\
	k^2 - \kappa^2 = \mu \epsilon w^2 ; 2 k \kappa = \mu \sigma w \nonumber \\
	\text{por lo que } \kappa=\frac{\mu \epsilon w}{2 k}\nonumber \\
\text{Metiendo las equivalencias obtenidas a las ecuaciones originales } \\ 
	k^4 - \frac{\mu^2 \sigma^2 w^2}{4} - \mu \epsilon w^2=0 \nonumber\\
	\text{Completando cuadrados llegamos a } \nonumber \\
	 k= w \sqrt{\frac{\epsilon \ mu}{2}} [\sqrt{1+(\frac{\sigma}{\epsilon w})^2}+1]^\frac{1}{2} \\
	 \text{como } \kappa^2 = k^2 - \mu \epsilon w^2 \text{luego} \nonumber\\
	 \kappa = w \sqrt{\frac{\epsilon \ mu}{2}} [\sqrt{1+(\frac{\sigma}{\epsilon w})^2}-1]^\frac{1}{2}
	\end{gather}
\end{subequations}
%%%%%%%%%%%%%%%%%%%%%%%%%%%%%%%%%%%%%%%%%%%%%%%%%%%%%%%%%%%%%%%%%%%%%%%%%%%%%%%%%%%%%%%%%%%%%%%%%%
\subsubsection{Guía cuadrada}
\begin{subequations}
	\begin{align}
		B_{z}(x,y)=X(x)Y(y) \nonumber \\
		Y d_{x}^2+X d_{y}^2Y + [(\frac{w}{v})^2-k^2]XY=0 \nonumber \\
		\frac{}{x} dx_{x}^2=-k_{x}^2 ; \frac{}{Y}d_{y}^2 Y = -K_{y}^2 \nonumber \\
		-k_{x}^2 - k_{y}^2+(\frac{w}{v})^2 - k^2 = 0 \nonumber\\
		\text{Proponiendo } \nonumber \\
		X(x)=A sin (K_{x} x) + B cos(k_{x}x)  \\
		\text{metendielo en las ecuaciones originales} \nonumber \\
		B_{z}=B_{0} cos(m \pi \frac{x}{a}) cos(n \pi \frac{y}{a}) \\
		k = \sqrt{(\frac{w}{v})^2-\pi^2 [(\frac{m}{a})^2+(\frac{n}{b})^2]} 
	\end{align}
\end{subequations}
%%%%%%%%%%%%%%%%%%%%%%%%%%%%%%%%%%%%%%%%%%%%%%%%%%%%%%%%%%%%%%%%%%%%%%%%%%%%%%%%%%%%%%%%%%%%%%%%%%
\subsubsection{Guías esféricas}

%%%%%%%%%%%%%%%%%%%%%%%%%%%%%%%%%%%%%%%%%%%%%%%%%%%%%%%%%%%%%%%%%%%%%%%%%%%%%%%%%%%%%%%%%%%%%%%%%%
\subsection{Guías dieléctricas}
Que son las ecuaciones de Helmholtz en dos dimensiones. Estas ecuaciones pueden ser generalizadas a
\begin{subequations}
	\begin{align}
	B_{t}=\frac{}{\mu \epsilon\frac{w^2}{c^2}-k^2}[D_{t} (d_{z} B{z})+i\mu\epsilon\frac{w}{v}e_{3}\wedge D_{t}E_{z} \\
	E_{t}=\frac{}{\mu \epsilon\frac{w^2}{c^2}-k^2}[D_{t} (d_{z} E{z})-i\frac{w}{v}e_{3}\wedge D_{t}B_{z} 
	\end{align}
	\end{subequations}
Donde $t$ es la componente transversal